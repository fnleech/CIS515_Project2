\documentclass[12pt]{article}
\usepackage{amsfonts,amssymb,amsmath,epsfig}
%\documentstyle[12pt,amsfonts]{article}
%\documentstyle{article}

\setlength{\topmargin}{-.5in}
\setlength{\oddsidemargin}{0 in}
\setlength{\evensidemargin}{0 in}
\setlength{\textwidth}{6.5truein}
\setlength{\textheight}{8.5truein}
\setcounter{MaxMatrixCols}{15}
%\input ../basicmath/basicmathmac.tex
%
%\input ../adgeomcs/lamacb.tex
\input mac-new.tex
\input mathmac-v2.tex
%\input ../adgeomcs/mac.tex
%\input ../adgeomcs/mathmac.tex

\def\fseq#1#2{(#1_{#2})_{#2\geq 1}}
\def\fsseq#1#2#3{(#1_{#3(#2)})_{#2\geq 1}}
\def\qleq{\sqsubseteq}

%
\begin{document}
\begin{center}
\fbox{{\Large\bf Fall 2016 \hspace*{0.4cm} CIS 515}}\\
\vspace{1cm}
{\Large\bf Fundamentals of Linear Algebra and Optimization\\
Jean Gallier \\
\vspace{0.5cm}
Project 2}\\[10pt]
Due October 27, 2016\\
Frankie Leech, Chen Xiang, Reffat Manzur\\
\end{center}

\vspace{0.5cm}

\medskip
(4)
Write two {\tt Matlab} functions {\tt haar2D} and {\tt haar\_inv2D}
implementing the method for computing the Haar transform
of a  matrix and the reconstruction of an image from its matrix of Haar
coefficients, as described in the notes.\\------------------------------\\
According to the notes, the Haar transform of a matrix is given by $C = (W_m^{-1})^T A (W_n^{-1})^T$ and the reconstruction of an image from its Haar coefficients by $A = W_m C W_n^T$ . To get {\tt haar2D}, we can think of a matrix A as    composed of m row vectors (u) with n elements in each vector. Then, we can perform the averaging and differencing algorithm on each row as given by:
$$c^j(i) = (c^{j+1}(2i -1) + c^{j+1}(2i))/2$$  
$$c^j(2^j+i) = (c^{j+1}(2i -1) - c^{j+1}(2i))/2$$ 

Afterwards, we perform averaging and differencing algorithm on the columns of the matrix $B=A W_n^{-1}$.


For {\tt haar\_inv2D},we reconstruct an image from its matrix coefficients by using the algorithm: 
$$u^{j+1}(2i-1) = (u^{j}(i) + u^{j}(2^j+i))$$  
$$u^{j+1}(2i) = (u^{j}(i) - u^{j}(2^j+i))$$
Apply the function {\tt haar\_inv2D} to the matrix
\[
T = 
\begin{pmatrix}
1212 & -306 & -146 & -54 & -24 & -68 & -40 & 4 \\
    30 & 36 & -90 & -2 &  8 & -20 & 8  & -4 \\
    -50 & -10 & -20 & -24 & 0 & 72 & -16 & -16 \\
    82 & 38 & -24 & 68 & 48 & -64 & 32 & 8 \\
    8 & 8 & -32 & 16 & -48 & -48 & -16 & 16 \\
    20 & 20 & -56 & -16 & -16 & 32 & -16 & -16 \\
    -8 & 8 & -48 & 0 & -16 & -16 & -16 &  -16 \\
    44 & 36 & 0 & 8 & 80 & -16 & -16 & 0
\end{pmatrix}.
\]

The method we use to compute the Haar transform of matrix T














Compare your result with the matrix $P$ of Example 4.1 of the paper by
Greg Ames (see the web page for CIS515). The matrix in Ames's paper
seems to have at typo! What is it?

\medskip
You can load and display 
various images in {\tt Matlab} using the following lines of code:
\begin{verbatim}
clear X map
load('durer','X')
Xdurer = X(1:512,:);
Xdurer(:,510:512) = 50;
figure
colormap(gray)
imagesc(Xdurer)
\end{verbatim}
The above loads the file {\tt durer}.
There are a few other  images such as {\tt detail}, {\tt flujet}, {\tt earth},
{\tt mandrill}, {\tt spine}, and {\tt clown}. You may have to resize
these images to have dimensions that are powers of $2$.
To display an image, use {\tt imagesc}.

\medskip
Convert {\tt Xdurer}  to its Haar transform and decode it. Compare the
original and the reconstructed image.


\medskip
(5)
Write two {\tt Matlab} functions {\tt haar2D\_n} and {\tt haar\_inv2D\_n}
implementing the method for computing the normalized Haar transform
of a  matrix and the reconstruction of an image from its matrix of
normalized Haar
coefficients, as described in the notes.

\medskip
Consider the image given by the following matrix:
\[
A =
\begin{pmatrix}
100 & 103 & 99 & 97 & 93 & 94 & 78 & 73 \\
 102 & 97 & 100 & 111 & 113 & 104 & 96 & 82 \\
 99 & 109 & 104 & 95 & 93 & 92 & 88 & 76 \\
 114 & 104 &  99 & 102 & 93 & 82 & 74 & 74 \\
 96 & 91 & 91 & 87 & 79 & 78 & 77 & 76 \\
 90 & 88 & 83 & 78 & 77 & 74 & 76 & 76 \\
 92 & 81 & 73 & 72 & 69 & 65 & 66 & 62 \\
 75 & 70 & 69 & 65 & 60 & 55 & 61 & 65
\end{pmatrix}
\]

Use {\tt haar2D\_n} to compute the normalized matrix $C$ of Haar
coefficients of $A$.

\medskip
It is claimed in Ames's paper (Section 7) that the reconstructed  matrix
\[
A_2 =
\begin{pmatrix}
100  &  100  &   95  &   95  &   92  &   92  &   76  &   76 \\
   103 &   103  &   98  &   98  &  106  &  106  &   90  &   90 \\
    99  &  109  &   99  &   99  &   96  &   96  &   81  &   81 \\
   114  &  104  &  104  &  104  &   91  &   91  &   76  &   76 \\
    91  &   91  &   86  &   86  &   76  &   76  &   76  &   76 \\
    91  &   91  &   86  &   86  &   76  &   76  &   76  &   76 \\
    82  &   82  &   76  &   76  &   66  &   66  &   66  &   66 \\
    74  &   74  &   69  &   69  &   58  &   58  &   59  &  59
\end{pmatrix}
\]
is obtained from the normalized matrix
\[
C_1 =
\begin{pmatrix}
255  &  52  &   15  &  21  &  0  &  0  &  0  &  0 \\
78   &    0   &    0   &  22  &  0  &  0  &  0  &  0 \\
0     &    0   &    0   &    0  &  0  &  0  &  0  &  0 \\
38     &    0   &    0   &    0  &  0  &  0  &  0  &  0 \\
0     &   11  &    0   &    0  &  0  &  0  &  0  &  0 \\
0     &    0   &    0   &    0  &  0  &  0  &  0  &  0 \\
0     &    0   &    0   &    0  &  0  &  0  &  0  &  0 \\
15     &    0   &    0   &    0  &  0  &  0  &  0  &  0 
\end{pmatrix},
\]
but this not quite correct. First, the coefficient $255$ should be
$682$, and other nonzero entries are missing.
Find the matrix $C_2$, a compressed version of $C$, that gives back
$A_2$.

\medskip
\hint
The command {\tt round} is helpful.


\end{document}
